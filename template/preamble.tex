
%>>>>>>>>>>>>>>>>>> ПРЕАМБУЛА >>>>>>>>>>>>>>>>>>>>
\documentclass[14pt,final,oneside]{extreport}
%>>>>> Разметка документа
\usepackage[a4paper, mag=1000, left=3cm, right=1.5cm, top=2cm, bottom=2cm, headsep=0.7cm, footskip=1cm]{geometry} % По ГОСТу: left>=3cm, right=1cm, top=2cm, bottom=2cm,
\linespread{1} % межстройчный интервал по ГОСТу := 1.5
%<<<<< Разметка документа

%>>>>> babel c языковым пакетом НЕ должны быть первым импортируемым пакетом
\usepackage[utf8]{inputenc}
\usepackage[T1,T2A]{fontenc}
\usepackage[russian]{babel}
%<<<<<

%\usepackage{cmap} %поиск в pdf

%>>>...>> прочие полезные пакеты
\usepackage{amsmath,amsthm,amssymb}
\usepackage{mathtext}
\usepackage{indentfirst}
\usepackage{graphicx}
\graphicspath{{/home/ivan/itmo/informatics/latex}}
\DeclareGraphicsExtensions{.pdf,.png,.jpg}
%\usepackage{bookmark}

\usepackage[dvipsnames]{xcolor}
\usepackage{hyperref}  % Использование ссылок
\hypersetup{%  % Настройка разметки ссылок
    colorlinks=true,
    linkcolor=blue,
    filecolor=magenta,      
    urlcolor=magenta,
    %pdftitle={Overleaf Example},
    %pdfpagemode=FullScreen,
}

%>>>>> Использование листингов
\usepackage{listings} 
\usepackage{caption}
\DeclareCaptionFont{white}{\color{white}} 
\DeclareCaptionFormat{listing}{\colorbox{gray}{\parbox{\textwidth}{#1#2#3}}}

\captionsetup[lstlisting]{format=listing,labelfont=white,textfont=white} % Настройка вида описаний
\lstset{  % Настройки вида листинга
inputencoding=utf8, extendedchars=\true, keepspaces = true, % поддержка кириллицы и пробелов в комментариях
language=Pascal,            % выбор языка для подсветки (здесь это Pascal)
basicstyle=\small\sffamily, % размер и начертание шрифта для подсветки кода
numbers=left,               % где поставить нумерацию строк (слева\справа)
numberstyle=\tiny,          % размер шрифта для номеров строк
stepnumber=1,               % размер шага между двумя номерами строк
numbersep=5pt,              % как далеко отстоят номера строк от подсвечиваемого кода
backgroundcolor=\color{white}, % цвет фона подсветки - используем \usepackage{color}
showspaces=false,           % показывать или нет пробелы специальными отступами
showstringspaces=false,     % показывать илигнет пробелы в строках
showtabs=false,             % показывать или нет табуляцию в строках
frame=single,               % рисовать рамку вокруг кода
tabsize=2,                  % размер табуляции по умолчанию равен 2 пробелам
captionpos=t,               % позиция заголовка вверху [t] или внизу [b] 
breaklines=true,            % автоматически переносить строки (да\нет)
breakatwhitespace=false,    % переносить строки только если есть пробел
escapeinside={\%*}{*)}      % если нужно добавить комментарии в коде
}
%<<<<< Использование листингов

\sloppy % Решение проблем с переносами (с. 119 книга Львовского)
\emergencystretch=25pt


%>>>>>>>>>>>>>>>> КОМАНДЫ {Для соответствия ГОСТ} >>>>>>>>>>>>>>

\newcommand\Chapter[2]{
    % Принимает 2 аргумента - название главы и дополнительный заголовок 
    \refstepcounter{chapter}
    \chapter*{
        \begin{huge}
        % Отключена нумерация глав в тексте:
        %:=% \textbf{\chaptername\ \arabic{chapter}\\}
        \textbf{#1}
        \end{huge}
        \raggedright
        \bigskip \bigskip
        #2
    }
    % Отключена нумерация для chapter в toc (table of contents), т.е. Оглавлении (Содержании):
    %:=% \addcontentsline{toc}{chapter}{\arabic{chapter}. #1}
    % Представление главы в содержании:
    \addcontentsline{toc}{chapter}{#1 #2} 
}


\newcommand\Section[1]{
    % Принимает 1 аргумент - название секции
    \refstepcounter{section}
    \section*{%
        \raggedright
        % Отключена дополнительная нумерация chapter в section в тексте документа:
        %:=% \arabic{chapter}.\arabic{section}. #1}
        % Отключена любая нумарация section в тексте документа: (убрать \arabic{section}, оставить название секции)
        \arabic{section}. #1
    }
    
    % Отключена дополнительная нумерация chapter в section в toc (table of contents) Оглавлении (Содержании):
    %:=% \addcontentsline{toc}{section}{\arabic{chapter}.\arabic{section}. #1}
    \addcontentsline{toc}{section}{\arabic{section}. #1} 
}


\newcommand\Subsection[1]{
    % Принимает 1 аргумент - название подсекции
    \refstepcounter{subsection}
    \subsection*{%
        \raggedright%
        % Отключена дополнительная нумерация chapter в section в тексте документа (можно добавить отступ с помощью \hspace*{12pt}):
        %:=% \arabic{chapter}.\arabic{section}.\arabic{subsection}. #1}
        \arabic{section}. \arabic{subsection}. #1
    }
    % Отключена дополнительная нумерация chapter в section в Оглавлении (Содержании):
    %\addcontentsline{toc}{subsection}{\arabic{chapter}.\arabic{section}.\arabic{subsection}. #1}
    \addcontentsline{toc}{subsection}{\arabic{subsection}. #1}
}


\newcommand\Figure[4]{
    % Принимает 4 аргумента - название файла изображения, ее размер в тексте, описание, лэйбл (псевдоним) 
        \refstepcounter{figure}
        \begin{figure}[h]
            \center{\includegraphics[width=#2]{#1}}
        \caption{#3}
        \label{fig:#4}
    \end{figure}
}

%<<<<<<<<<<<<<<<<<<<<<<<<<<<< КОМАНДЫ <<<<<<<<<<<<<<<<<<<<<<<<<<
